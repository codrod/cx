\hypertarget{_2home_2tux_2dev_2cx_2examples_2exceptions_8c-example}{}\section{/home/tux/dev/cx/examples/exceptions.\+c}
Example of exception handling.


\begin{DoxyCodeInclude}

\textcolor{preprocessor}{#include <\hyperlink{cxlib_8h}{cxlib.h}>}

\textcolor{keywordtype}{int} func();

\textcolor{keywordtype}{int} main()
\{
    \_\_cxmain\_\_

    cxtry
    \{
        \textcolor{comment}{//All CX exceptions are derived from CXException}
        cxthrow(CXException);
    \}
    cxcatch()
    \{
        \textcolor{comment}{//cxexcept is the excpetion caught above}
        printf(\textcolor{stringliteral}{"caught %s\(\backslash\)n"}, cxexcept.type);

        \textcolor{comment}{//continues}
    \}

    cxtry
    \{
        cxbyte\_t *p = NULL;

        cxbyte\_t b = *p;
    \}
    \textcolor{comment}{//Multiple exceptions can be caught including seg faults}
    cxcatch(CXException\_Signal\_SegFault, CXException\_Memory\_NoMemory)
    \{
        printf(\textcolor{stringliteral}{"caught %s\(\backslash\)n"}, cxexcept.type);
    \}

    cxtry
    \{
        func();
    \}
    cxcatch(CXException\_Memory\_NoMemory)
    \{
        printf(\textcolor{stringliteral}{"caught %s\(\backslash\)n"}, cxexcept.type);
    \}

    cxreturn 0;
\}

\textcolor{keywordtype}{int} func()
\{
    \textcolor{comment}{//Initialize function}
    \_\_cxfunc\_\_

    cxthrow(CXException\_Memory\_NoMemory);

    \textcolor{comment}{//Dont forget to use cxreturn not return}
    cxreturn 0;
\}
\end{DoxyCodeInclude}
 