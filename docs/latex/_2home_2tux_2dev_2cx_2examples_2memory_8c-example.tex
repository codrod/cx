\hypertarget{_2home_2tux_2dev_2cx_2examples_2memory_8c-example}{}\section{/home/tux/dev/cx/examples/memory.\+c}
A hello world example.


\begin{DoxyCodeInclude}

\textcolor{preprocessor}{#include <\hyperlink{cxlib_8h}{cxlib.h}>}

cxbyte\_t* func();

\textcolor{keywordtype}{int} main()
\{
    \textcolor{comment}{//Initialize the runtime}
    \_\_cxmain\_\_

    \textcolor{comment}{//All memory stored in cxauto is deallocated on the next cxreturn}
    cxaddress\_t addr = cxmalloc(cxauto, 1000);

    \textcolor{comment}{//But memory can be moved to another storage after allocation.}
    \textcolor{comment}{//All memory stored in cxstatic is deallocated when the program exits}
    cxrestore(cxstatic, addr);

    \textcolor{comment}{//Note that addr may come from malloc or cxmalloc}
    cxfree(addr);

    printf(\textcolor{stringliteral}{"%p\(\backslash\)n"}, func());

    \textcolor{comment}{//You must use cxreturn not return}
    cxreturn 0;
\}

cxbyte\_t* func()
\{
    \_\_cxfunc\_\_

    \textcolor{comment}{//We can also return function local memory by pushing it before returning}
    cxbyte\_t* addr = cxpush(cxmalloc(cxauto, 1000));

    cxreturn addr;
\}
\end{DoxyCodeInclude}
 